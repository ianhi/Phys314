\documentclass[11pt,letterpaper]{article}

%%%%%%%%%%%%%%%%%%%%%%%%%%%%%%%%%%%%%%%%%%%%%%%%%%%%%%%%%%%%%%%%%%%%%%%%%
\pagestyle{plain}                                                      %%
%%%%%%%%%% EXACT 1in MARGINS %%%%%%%                                   %%
\setlength{\textwidth}{6.5in}     %%                                   %%
\setlength{\oddsidemargin}{0in}   %% (It is recommended that you       %%
\setlength{\evensidemargin}{0in}  %%  not change these parameters,     %%
\setlength{\textheight}{8.5in}    %%  at the risk of having your       %%
\setlength{\topmargin}{0in}       %%  proposal dismissed on the basis  %%
\setlength{\headheight}{0in}      %%  of incorrect formatting!!!)      %%
\setlength{\headsep}{0in}         %%                                   %%
\setlength{\footskip}{.5in}       %%                                   %%
%%%%%%%%%%%%%%%%%%%%%%%%%%%%%%%%%%%%                                   %%
%\newcommand{\required}[1]{\section*{\hfil #1\hfil}}                    %%
\renewcommand{\refname}{\hfil References Cited\hfil}                   %%
%\bibliographystyle{plain}                                              %%
%%%%%%%%%%%%%%%%%%%%%%%%%%%%%%%%%%%%%%%%%%%%%%%%%%%%%%%%%%%%%%%%%%%%%%%%%

%PUT YOUR MACROS HERE

\usepackage{amsmath}
\usepackage{amsfonts}
\usepackage{amssymb}
\usepackage{graphicx}
\usepackage{ulem}
\usepackage[hidelinks]{hyperref}


\title{NMR}


\author{Ian Hunt-Isaak\\ \begin{small}
Partner: Pete Sinn
\end{small}}

\begin{document}


\date{}
\maketitle
\section{} %1

Values of T=293 K, B$_0=3.53\pm .02$ kG gives us:
\begin{equation}
\Delta U = 8.46\cdot 10^{-27}
\end{equation}
\begin{equation}
\frac{N_2}{N_1} = e^{-\frac{\Delta E}{k_b * T}}=0.999 \pm 1.4\cdot 10^{-8}
\end{equation}
%\begin{equation*}
%100\cdot (1-\frac{N_2}{N_1})=0.000246 \pm 1.3 e-06 
%\end{equation*}
Thus .000246 $\pm$ 1.3 e-06 percent of the spins are sensitive to the pulsed NMR technique. If the signal was poor temperature could be reduced or the magnetic field increased in order to sensitize more spins. A reduction in temperature to 10 K brings the percentage of sensitive spins up to .007 \% with a $10^{1.5}$ times increase in the magnetic field bringing a similar increase. Both of these push on the limits of what is reasonably achievable so the best avenue for improvement may be more sensitive detector equipment. 
\section{} %2
From the frequency of the rf field the dc field was calculated to have a value of 3.569 $\pm 9.39e-6$ kG. This is within 2$\sigma$ of the value obtained with the Gaussmeter: 3.53 $\pm$ .02 kG. The uncertainty in the calculated value was propagated from the uncertainty in the resonance frequency, however this was a very small percent error resulting in the small error. The calculated value is a more accurate value as it averages the field value over a larger space than the Gaussmeter and thus deals more fully with field inhomogeneities. The small discrepancy between the measured and calculated values can be explained by the difficulty in deciding where to place the Gaussmeter coupled with the fact the Gaussmeter only captures a section of the space.


\section{} %3
\label{sec:q3}
The differential equation describing the Z axis magnetization under the influence of an external field applied along the axis is:
\begin{equation}
\frac{dM_z}{dt}=\frac{M_0 - M_Z}{T_1}
\end{equation}
Solving for M$_Z$(t)results in:
\begin{equation}
M_Z(t) = M_0-Ce^{\frac{-t}{T_1}}, \ \ C\in \mathbb{R}
\end{equation}
In the two pulse technique the first pulse, a $\pi$ pulse, sets the initial conditions of the system to 
\begin{align*}
t=0, M_z(0)=-M_0
\end{align*} giving us:
\begin{equation}
\label{eq:Mz}
M_Z(t) = M_0(1-2e^{\frac{-t}{T_1}})
\end{equation}
The second pulse, $\frac{\pi}{2}$, some time $\tau$ later will then flip the spins into the transverse plan, allowing us to record the absolute value of M$_Z$. From the fact the M$_Z$(t) is monotonically increasing we deduce that the concave up portion of the data corresponds to negative M$_Z$ values. Thus by inverting the sign of these data we obtain a M$_Z$ curve which we fit to \ref{eq:Mz} in order to obtain T$_2$.


By quick estimation we found that T$_1=1$ ms, through fitting we found that T$_1=   $ ms. Our estimate was on the same order of magnitude as the value from the fit. That we were not more accurate in our estimate is reasonable as the dial for repetition time only had order of magnitude and there are approximations involved in the quick estimate.




\section{} %4



T$_2$ was determined by fitting the $\Delta$V between the peaks in from the spin echo technique to:

\begin{equation}
\Delta V=M_0(1-e^{\frac{-t}{T_2})}
\end{equation}

T$_1$ was calculated as described in \ref*{sec:q3}. Values are below.

\begin{center}
	\begin{tabular}{|c|c|}\hline
		Quantity & Value \\ \hline\hline
		T$_2^*$ & .47 ms \\ \hline
		T$_2$ & 6.25 ms \\ \hline
		T$_1$ & 21.69 ms \\ \hline
	\end{tabular}
\end{center}
As expected T$_2^*$ is significantly smaller than T$_2$. Also note that T$_1$ is $\sim$3x more than T$_2$. This is sensible as T$_1$ is associated with the realignment of the spins anti-aligned to the external field while T$_2$ corresponds to the realignment of spins which are perpendicular to the field. So T$_1$ corresponds to a much more dramatic realignment and also involves transfer of energy to the lattice.

\subsection{Uncertainty Budget}
FINISH THIS \\
\begin{table}[h]
	\begin{center}
		\begin{tabular}{|c|c|c|c|} \hline 
			Source & Quantity&  Error in Quantity  & Propagated Error  \\ \hline \hline
			Statistical Error & T$_1$ & .35 (ms) & .35 (ms)\\ \hline 
			Statistical Error & T$_2$ & .6 (ms)& .6 (ms)\\ \hline
			Pulse Width& $\tau$ & 5\% & Negligible\\
			\hline
		\end{tabular}
	\end{center}
\end{table}

Due to the use of cursors to determine the Voltages for determining T$_1$ and T$_2$ there is error in the fits. We estimate the error in the voltage values as $\pm .05$V. This combined with statistical errors in fitting is the dominant source of uncertainty.

The other source of the uncertainty was the the inability of to produce an exact $\frac{\pi}{2}$ pulse. However this is ultimately a negligible effect as the probability of excitation is described by a sin curve and a $\frac{\pi}{2}$ pulse is located at an anti-node of this curve and so a small deviation in pulse width will not have a dramatic effect.

The $\chi^2$ values are included on the plots and are do not discount our model.
\section{} %5
Field inhomogeneities cause spins to precess at different rates leading to a loss of signal in the pickup coil due to the spins decohering. If this rate of decoherence, T$_2$*,  is short compared to the quantity T$_2$ a measurement of the FID will be a measurement of T$_2$* as the T$_2$ processes will not have had time to become significant. 

The two-pulse spin echo technique avoids this issue by forcing the spins to recohere on the T$_2$ time scale. This adds a rise and subsequent drop in signal some time after the initial FID. Both drops in signal with this technique correspond to decoherence of spins however the second peak will have lower maximum as during the delay between the peaks some spins will have decayed due to T$_2$ processes. The difference between the peaks will thus correspond to the number of spins that have relaxed back to the Z axis due to T$_2$ processes.
\end{document}
\documentclass[12pt,a4paper]{article}
\usepackage[latin1]{inputenc}
\usepackage{amsmath}
\usepackage{amsfonts}
\usepackage{amssymb}
\usepackage{graphicx}
\usepackage[margin=1in]{geometry}

\title{Vacuum Systems Techniques} % Title

\author{Ian \textsc{Hunt-Isaak}} % Author name



\begin{document}

\maketitle % Insert the title, author and date

\begin{center}
\begin{tabular}{l r}
Date Started: & September 8, 2015 \\ % Date the experiment was performed
Partner: & Jonah Glasgold \\% Partner names
\end{tabular}
\end{center}


\section{}
We did not evidence of a significant quantity of condensable vapor in the system. Had we found any it most likely would have been gases which had been adsorbed onto the various surfaces inside the bell jar. The purpose of baking the vacuum system is to prevent vapors from condensing on surfaces and to release gases that had been adsorbed. 
\section{}
The leak that we found was a single hair stuck under the bell jars rubber lining. This was found using the MS-40 leak detector however this leak could have been detected through careful visual inspection. We were not accustomed to checking vacuum systems for leaks and therefore used the MS-40 leak detector as a visual inspection missed the hair. 

We managed to isolate the leak by moving a nozzle emitting a very light spray of Helium around suspect areas and listening to the tones output by the leak detector to determine the general location of the leak. Once the leak was located the hair causing the leak was immediately obvious. 

The leak detector isolates leaks through attachment of its mass spectrometer to the vacuum system. Helium is then sprayed around suspect areas and if there is a leak the Helium will enter the vacuum system and reach the mass spectrometer, the MS-40 will then register the change in the quantity of Helium and change its output tone to denote that Helium has entered the system.
\section{}
The ultimate pressure limit is set by the vapor pressure of the pump and the effectiveness of the cold baffle used to prevent diffusion pump oil from backstreaming into the system.

\section{}
\subsection*{Pumps}
All three types of pumps we used are similar in the basic principle of operation which is that gas molecules in the system are somehow forced from the low pressure region to a higher pressure region. The Turbomolecular and Diffusion pumps do this in a similar manner by transferring momentum either with a jet of fast moving molecules (diffusion pump) or turbines which force any molecules down. The mechanical pump traps whatever molecules have diffused into it and then pushes them into a higher pressure volume through a rotary motion. The Mechanical rotary pump differs from the other two pump types in that it can operate at atmospheric pressure. Both the Turbomolecular and Diffusion pump have upper pressure limits above which they will not operate or break.
\subsection*{Gauges}
Pirani and Ionization Gauges are similar in that they both measure pressure and have pressure ranges in which they can operate. They are however very different in both their method of measuring pressure and operating ranges. Ionization gauges work through a triode set up in which electrons are blocked more at higher pressure whereas a pirani gauge works through detecting the rate of heat loss due to collisions with air molecules. The Pirani gauge is limited to higher pressures than the ionization gauge as at low enough pressures the major source of heat dissipation is radiative. The ionization gauge is held to lower pressures because at higher pressure values there are too many gas molecules for electrons to make it across the triode and so none can be detected.
\section{}
The plot of the data was fairly curved, in contrast to the linear plot predicted by our model. The data is emphatically not perfectly linear, causing our result to depend on the region of the plot selected for the linear fit. For our calculations we used the higher pressure regions of the data as the data was more linear there than at lower pressure regions. The data not being linear shows a failure of our model. 

Our data was different from run to run with the first two runs being far less linear than the third run. Due to the behavior of system during pumping down we suspect that there was a leak during the first two runs and so did not use those runs in our analysis.
\section{}
Our theoretical value of $470\ m/s$ is several standard deviations away from our experimental value of $420\ \pm 4\ m/s$. The theoretical root mean square velocity is $730\ m/s$. The rms value is much larger than our experimental average speed, this is to be expected however as the rms should be larger than the average.


The largest source of uncertainty in our final results was the calculation of the area of the hole. Other sources of error were the volume of the chamber and significantly the model uncertainty as our model was clearly flawed. The two main things to be improved upon are the model itself and the measurement of the area of the hole. 

\end{document}
%%%%%%%%%%%%%%%%%%%%%%%%%%%%%%%%%%%%%%%%%%%%%%%%%%%%%%%%%%%%%%%%%%%%%%
% Overleaf (WriteLaTeX) Example: Molecular Chemistry Presentation
%
% Source: http://www.overleaf.com
%
% In these slides we show how Overleaf can be used with standard 
% chemistry packages to easily create professional presentations.
% 
% Feel free to distribute this example, but please keep the referral
% to overleaf.com
% 
%%%%%%%%%%%%%%%%%%%%%%%%%%%%%%%%%%%%%%%%%%%%%%%%%%%%%%%%%%%%%%%%%%%%%%
% How to use Overleaf: 
%
% You edit the source code here on the left, and the preview on the
% right shows you the result within a few seconds.
%
% Bookmark this page and share the URL with your co-authors. They can
% edit at the same time!
%
% You can upload figures, bibliographies, custom classes and
% styles using the files menu.
%
% If you're new to LaTeX, the wikibook is a great place to start:
% http://en.wikibooks.org/wiki/LaTeX
%
%%%%%%%%%%%%%%%%%%%%%%%%%%%%%%%%%%%%%%%%%%%%%%%%%%%%%%%%%%%%%%%%%%%%%%

\documentclass{beamer}

% For more themes, color themes and font themes, see:
% http://deic.uab.es/~i %lanes/beamer_gallery/index_by_theme.html
%
\mode<presentation>
{
  \usetheme{CambridgeUS}       % or try default, Darmstadt, Warsaw, ...
  \usecolortheme{seahorse} % or try albatross, beaver, crane, ...
  \usefonttheme{default}    % or try default, structurebold, ...
  \setbeamertemplate{navigation symbols}{}
  \setbeamertemplate{caption}[numbered]
  \setbeamerfont{frametitle}{size=\small}
  \setbeamertemplate{bibliography item}[text]
} 

\usepackage[english]{babel}
%\usepackage[utf8x]{inputenc}

% Here's where the presentation starts, with the info for the title slide
\title[Muon Rate - Phys 314]{Time Variation of Muon Rate}
\author{Ian H-I}
\institute{}
\date{December 10, 2015}

\begin{document}

\begin{frame}
  \titlepage
\end{frame}

% These three lines create an automatically generated table of contents.
\begin{frame}{Outline}
  \tableofcontents
\end{frame}

\section{Introduction}

\begin{frame}{Introduction}

\begin{itemize}
\item MUONS N SHIT
\end{itemize}


\end{frame}

\subsection{The Experiment}

\begin{frame}{Block Diagram}

\begin{block}{Sweet Boundaries}
God this text has a beautiful boundary
\end{block}


\end{frame}
\subsection{Theory}
\begin{frame}{Cosmic Rays}
sdfla
\begin{block}{ERTH}
It turns out the earth rotates
\end{block}

\end{frame}
\begin{frame}{Electronics and Temp}
sdfla
\end{frame}


\section{Using chemistry packages with \LaTeX{}}

\subsection{No one likes chemistry}

\begin{frame}[fragile]
\frametitle{Chemical equations with \texttt{mhchem}}

\begin{itemize}
\item The \texttt{mhchem} package lets you write chemical equations in \LaTeX{} with the minimum of effort. 
\item The example below shows how the standard representation of a reaction (on the left) is created from the simple code on the right:
\end{itemize}


\begin{itemize}
\item More complicated reactions are still easy to write:
\end{itemize}

\end{frame}

\subsection{Getting started with some \texttt{chemfig} coffee}

\begin{frame}[fragile]
\frametitle{Getting started with some \texttt{chemfig} coffee}

It's easy to use the \texttt{chemfig} package for drawing complex molecules:

\vskip 0.5cm


If that looks quite daunting, we can learn from simpler molecules\dots{}how about a single water molecule?

\end{frame}

\subsection{Experiments with water and rings}

\begin{frame}[fragile]
\frametitle{Experiments with water and rings}

To see how the \texttt{chemfig} package creates the drawings from your code, let us look at the simple water molecule:


The simple \LaTeX{} code on the right is automatically converted into the molecular formula for water on the left. 
\vskip 0.3cm
Rings are similarly easy to code - consider the examples below:

\vskip 0.3cm

\end{frame}

\section{Where to go next\dots{}}

\begin{frame}{Where to go next\dots{}}

\begin{itemize}
\item This short example was designed to introduce you to using Overleaf for scientific presentations.
\item This is made possible by the many great packages that have been developed for \LaTeX{}, including the two we focused on here (plus the \texttt{Beamer} package used for the overall presentation style). 
\item For more help on using \LaTeX{}, see the links on the Overleaf help page: 
\end{itemize}


\end{frame}
\begin{frame}
\begin{thebibliography}{9}

\bibitem{lamport94}
  Leslie Lamport,
  \emph{\LaTeX: a document preparation system},
  Addison Wesley, Massachusetts,
  2nd edition,
  1994.

\end{thebibliography}
\end{frame}

\end{document}